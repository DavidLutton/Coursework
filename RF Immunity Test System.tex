\documentclass[a4paper]{article}
\usepackage[utf8]{inputenc}
\usepackage{import}
%\usepackage{showframe}

\usepackage{listings}
\usepackage{color}

\definecolor{codegreen}{rgb}{0,0.6,0}
\definecolor{codegray}{rgb}{0.5,0.5,0.5}
\definecolor{codepurple}{rgb}{0.58,0,0.82}
\definecolor{backcolour}{rgb}{0.95,0.95,0.92}

\lstdefinestyle{mystyle}{
    backgroundcolor=\color{backcolour},
    commentstyle=\color{codegreen},
    keywordstyle=\color{magenta},
    numberstyle=\tiny\color{codegray},
    stringstyle=\color{codepurple},
    basicstyle=\ttfamily\bfseries,
    breakatwhitespace=false,
    breaklines=true,
    captionpos=b,
    keepspaces=true,
    numbers=left,
    numbersep=5pt,
    showspaces=false,
    showstringspaces=false,
    showtabs=false,
    tabsize=2
}
%\footnotesize
\lstset{style=mystyle}


% \usepackage{minted}
% \usepackage{tracklang}
\usepackage{amsmath}
\usepackage[tableposition=top]{caption}

\usepackage{color}   %May be necessary if you want to colour links
\usepackage{hyperref}
\hypersetup{
    colorlinks=true, %set true if you want colored links
    linktoc=all,     %set to all if you want both sections and subsections linked
    linkcolor=blue,  %choose some color if you want links to stand out
    linktocpage=true
}
\usepackage[toc, nopostdot, nonumberlist]{glossaries} % sort=def,
\usepackage[top=20mm, bottom=20mm, left=30mm, right=30mm]{geometry}
\usepackage[english]{babel}
\usepackage{fancyhdr}

\pagestyle{fancy}
\fancyhf{}



\newglossaryentry{CDN} {
  name={CDN},
  description={Coupling Decoupling Network: Interface in-between a power or signal line and injected RF},
}
\newglossaryentry{dBm} {
  name={dBm},
  description={ \= 10 * log10(power / 0.001) Power referenced to 1mW}
}
\newglossaryentry{EUT} {
  name={EUT},
  description={Equipment Under Test}
}

\newglossaryentry{DUT} {
  name={DUT},
  description={Device Under Test}
}

\newglossaryentry{AM} {
  name={AM},
  description={Amplitude modulation}
}

\newglossaryentry{Python} {
  name={Python},
  description={\href{https://www.python.org/}{Link} programming language \hyperref[sec:Python]{Section}}
}

\newglossaryentry{Git} {
  name={Git},
  description={\href{https://git-scm.com}{Link} distributed version control system}
}

\newglossaryentry{venv} {
  name={venv},
  description={\href{https://docs.python.org/3/library/venv.html}{Link} virtual environments for Python programs}
}

\newglossaryentry{pip} {
  name={pip},
  description={\href{https://pypi.python.org/pypi/pip}{Link} Manages module installation}
}

\newglossaryentry{Pypi} {
  name={Pypi},
  description={\href{https://pypi.python.org}{Link} Python package index}
}
\newglossaryentry{ISO8601} {
  name={ISO8601},
  description={\href{https://en.wikipedia.org/wiki/ISO_8601}{Link} Representation of dates and times, 2016-11-25T16:20:50+00:00}
}
\newglossaryentry{EMC} {
  name={EMC},
  description={Electromagnetic compatibility}
}
\newglossaryentry{EN61000-4-6:2014} {
  name={EN61000-4-6:2014},
  description={EMC Testing and measurement techniques. Immunity to conducted disturbances, induced by radio-frequency fields}
}

\newglossaryentry{RFC2119} {
  name={RFC2119},
  description={IETF Key words for use in RFCs to Indicate Requirement Levels \hyperref[sec:RFC2119]{Extract}}
}

\title{RF Immunity Test System}
\rhead{RF Immunity Test System}

\author{David Lutton}
\lhead{David Lutton}

\rfoot{Page \thepage}
\date{\today}
\makeglossaries

\begin{document}
\maketitle\tableofcontents
\newpage\begin{abstract}
RF Immunity testing is a significant segment of \gls{EMC} testing\\

Limitations of existing software include constraints on hardware it will execute on\\

This project is to replace and then extend the capabilities of immunity testing software\\

\end{abstract}

\section{Modus operandi}
Dates will be recorded in \gls{ISO8601} format\\
YYYY-MM-DDTHH:MM:ss+TZ:TZ\\
2016-11-25T16:20:50+00:00\\
The result of this every folder of name ordered files will be in date order\\
\\
Version control will be using the \gls{Git} version control system\\
\\
The specification MUST be defined by \gls{RFC2119}


\section{Initial Project Ideas : 2016/09/03}
I have been advised to for our initial ideas in a mind map\\
TODO: Attach IMAGE OF Mind map\\

\section{Engineering Project : Requirements : 2016/09/20}
The project consists or both project management and communications\\
This will require:\\
A logbook will all entries dated\\
Project specification\\
Logbook will itemise all communications:
\begin{itemize}
  \item Initial ideas and justification of project
  \item Specifications
  \item Technical drawings
  \item schematic
  \item calculations
  \item component specification
  \item simulation
  \item Planning
  \item Decisions
  \item Customer Communications
  \item Websites used
  \item decision matrix for solutions
  \item SWOT analysis
  \item GANTT chart for planning / time-line
\end{itemize}


Have processes of

\begin{itemize}
  \item Specify
  \item design
  \item Build
  \item Test
  \item Modify
  \item Evaluate
\end{itemize}

\section{Project : Selection Critical : 2016/09/27}
Should:\\
Provide something\\
Have design choices\\
be achievable (could)\\
Possibly as I do in my workplace\\
Could use a project in work for project\\
 Where naturally accessible at work\\

\section{Project : Selection Criteria : 2016/10/04}
Could:\\
A physical product\\
A system product, a remote monitoring system\\
A service product, a engineering service\\
\\
Replication or extension of an existing product to enhance the service\\

\section{Project Selection}
TODO:\\

\newpage

\section{Project Final Idea : 2016/10/11}
RF Immunity test system\\
\begin{itemize}
  \item Control test equipment
  \item Monitor and control via feedback path
  \item Produce test report data
  \item Use Calibration factors
  \item Apply test standard \gls{EN61000-4-6:2014}
\end{itemize}
Customers will be test engineers at work\\
TODO: embed TestSystemNetwork.xlsx  = \\
Diagram showing Normal testing set-up and calibration set-up\\

\section{Needs analysis : 2016/10/18}
To access is there a commercial or technical market\\
A survey may be used to derive a set of design specifications\\
product performs in a repeatable manner\\
product uses proven technology\\
low incremental running cost : per runtime\\
simple to maintain\\
adaptable to new tests\\
must stop/abort a test in a safe manner\\
must apply correction factors\\

Project Specification from customer:\\
Time scale\\
constraint dates\\
availability of test set up for development\\
plan installation\\
set up drawing / work instructions\\

\section{Select solution : 2016/10/25}
select from three solutions
This business sector has a couple of ready made platform products\\
LabVIEW by NI\\
VEE Pro by Keysight\\
Python 3 by Python Foundation\\
  With addition of some modules to communicate with test equipment\\

\newpage
\section{Test process : Calculations 2016/11/01}
Calibrations are performed a higher amplitude than the test level we need to calculate the level that we need to apply\\
Calibration\\
The table \ref{table:1} is an example of referenced \LaTeX elements.\\

\begin{table}[h!]
\centering
\begin{tabular}{||c c c c||}
 \hline
 Frequency Hz & level V & siggen dBm & forward power dBm \\ [0.5ex]
 \hline
 150e3 & 18 & -9 & 38.64 \\
 200e3 & 18 & -9 & 38.46 \\
 \hline
\end{tabular}
\caption{Example Calibration data}
\label{table:1}
\end{table}
..\\
For a calibration performed at 18V\\
For a test run to be performed at 10V\\
Offset V = Cal power V : Wanted power V\\
Offset dBm = V2dBm(offset dBm)\\
Target forward power = forward power dBm : Offset dBm\\

Levelling performed without and with any AM applied\\
Dwell for time in both CW and AM modulation?\\

\section{Estimated time : 2016/11/08}
TODO:

\section{Unit conversion : 2016/11/08}
TODO:

\section{Project Specification Revised : 2016/11/15}
TODO:

\section{Ideal}
It should not be possible to encounter a bug, without also being able to reproduce a bugs state. In order to fix it.\\
Furthermore this system should sink it's state when encountering a exception\\

%\renewcommand{\ttdefault}{pcr}
\begin{lstlisting}
foo = False
\end{lstlisting}

\lstinputlisting[language=Python, caption=Python example]{x.py}





\newpage\section{About Python}
\label{sec:Python}
\gls{Python} is a programming language available from \href{https://python.org}{Python.org}\\

Windows XP support ended with 3.4.4\\
Otherwise always use the highest numbered version\\

When installing Python offers to include the runtime in PATH always set this option on\\
Otherwise you need to invoke the Python runtime with the full path every time\\

Extra Python functionality is built up of modules available from \gls{Pypi} or \gls{Git}\\
These modules are installable via \gls{pip}\\

For example:\\
\`pip install pyserial\`\\
Installs the module to utilise serial ports and includes a small serial terminal application.\\

\gls{venv} is a module that is part of the Python core that creates Python virtual environments\\
This venv can be used to install modules separately from the system installation\\
If used correctly this allows programs to be reliably installed and this provides reasonable assurance that the program will operate the same way on a different computer\\

Python is a programming language that uses tabs to delineate nested flow control\\

I have written a number of .bat scripts to install, start, run, (Utilities, Serial console, GPIB console, Code checkers)\\

\newpage\subsection{RFC2119}
\label{sec:RFC2119}
\gls{RFC2119} is a available from \href{https://tools.ietf.org/html/rfc2119}{IETF.org}\\

The remainder of this section is a extract of \href{https://tools.ietf.org/html/rfc2119}{IETF.org RFC 2119}
By: S. Bradner, Harvard University, March 1997\\

Abstract\\

   In many standards track documents several words are used to signify
   the requirements in the specification.  These words are often
   capitalized.  This document defines these words as they should be
   interpreted in IETF documents.  Authors who follow these guidelines
   should incorporate this phrase near the beginning of their document:\\

      The key words "MUST", "MUST NOT", "REQUIRED", "SHALL", "SHALL
      NOT", "SHOULD", "SHOULD NOT", "RECOMMENDED",  "MAY", and
      "OPTIONAL" in this document are to be interpreted as described in
      RFC 2119.\\

   Note that the force of these words is modified by the requirement
   level of the document in which they are used.\\

1. MUST   This word, or the terms "REQUIRED" or "SHALL", mean that the
   definition is an absolute requirement of the specification.\\

2. MUST NOT   This phrase, or the phrase "SHALL NOT", mean that the
   definition is an absolute prohibition of the specification.\\

3. SHOULD   This word, or the adjective "RECOMMENDED", mean that there
   may exist valid reasons in particular circumstances to ignore a
   particular item, but the full implications must be understood and
   carefully weighed before choosing a different course.\\

4. SHOULD NOT   This phrase, or the phrase "NOT RECOMMENDED" mean that
   there may exist valid reasons in particular circumstances when the
   particular behavior is acceptable or even useful, but the full
   implications should be understood and the case carefully weighed
   before implementing any behavior described with this label.\\

5. MAY   This word, or the adjective "OPTIONAL", mean that an item is
  truly optional.  One vendor may choose to include the item because a
  particular marketplace requires it or because the vendor feels that
  it enhances the product while another vendor may omit the same item.
  An implementation which does not include a particular option MUST be
  prepared to interoperate with another implementation which does
  include the option, though perhaps with reduced functionality. In the
  same vein an implementation which does include a particular option
  MUST be prepared to interoperate with another implementation which
  does not include the option (except, of course, for the feature the
  option provides.)\\

6. Guidance in the use of these Imperatives\\

  Imperatives of the type defined in this memo must be used with care
  and sparingly.  In particular, they MUST only be used where it is
  actually required for interoperation or to limit behavior which has
  potential for causing harm (e.g., limiting retransmisssions)  For
  example, they must not be used to try to impose a particular method
  on implementors where the method is not required for
  interoperability.\\


\newpage\glsaddall\printglossaries
\newpage\listoftables\lstlistoflistings

\end{document}
