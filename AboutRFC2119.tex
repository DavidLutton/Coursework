\subsection{RFC2119}
\label{sec:RFC2119}
\gls{RFC2119} is a available from \href{https://tools.ietf.org/html/rfc2119}{IETF.org}\\

The remainder of this section is a extract of \href{https://tools.ietf.org/html/rfc2119}{IETF.org RFC 2119}
By: S. Bradner, Harvard University, March 1997\\

Abstract\\

   In many standards track documents several words are used to signify
   the requirements in the specification.  These words are often
   capitalized.  This document defines these words as they should be
   interpreted in IETF documents.  Authors who follow these guidelines
   should incorporate this phrase near the beginning of their document:\\

      The key words "MUST", "MUST NOT", "REQUIRED", "SHALL", "SHALL
      NOT", "SHOULD", "SHOULD NOT", "RECOMMENDED",  "MAY", and
      "OPTIONAL" in this document are to be interpreted as described in
      RFC 2119.\\

   Note that the force of these words is modified by the requirement
   level of the document in which they are used.\\

1. MUST   This word, or the terms "REQUIRED" or "SHALL", mean that the
   definition is an absolute requirement of the specification.\\

2. MUST NOT   This phrase, or the phrase "SHALL NOT", mean that the
   definition is an absolute prohibition of the specification.\\

3. SHOULD   This word, or the adjective "RECOMMENDED", mean that there
   may exist valid reasons in particular circumstances to ignore a
   particular item, but the full implications must be understood and
   carefully weighed before choosing a different course.\\

4. SHOULD NOT   This phrase, or the phrase "NOT RECOMMENDED" mean that
   there may exist valid reasons in particular circumstances when the
   particular behavior is acceptable or even useful, but the full
   implications should be understood and the case carefully weighed
   before implementing any behavior described with this label.\\

5. MAY   This word, or the adjective "OPTIONAL", mean that an item is
  truly optional.  One vendor may choose to include the item because a
  particular marketplace requires it or because the vendor feels that
  it enhances the product while another vendor may omit the same item.
  An implementation which does not include a particular option MUST be
  prepared to interoperate with another implementation which does
  include the option, though perhaps with reduced functionality. In the
  same vein an implementation which does include a particular option
  MUST be prepared to interoperate with another implementation which
  does not include the option (except, of course, for the feature the
  option provides.)\\

6. Guidance in the use of these Imperatives\\

  Imperatives of the type defined in this memo must be used with care
  and sparingly.  In particular, they MUST only be used where it is
  actually required for interoperation or to limit behavior which has
  potential for causing harm (e.g., limiting retransmisssions)  For
  example, they must not be used to try to impose a particular method
  on implementors where the method is not required for
  interoperability.\\
